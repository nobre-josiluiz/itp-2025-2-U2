\documentclass[a4paper, 12pt]{report}

\usepackage[top=2.5cm, bottom=2.5cm, left=2.5cm, right=2.5cm, marginparwidth=2cm]{geometry}

\usepackage{setspace}
\onehalfspacing


\usepackage[utf8]{inputenc}
\usepackage[portuguese]{babel}
\usepackage{graphicx} 
\usepackage{hyperref} 


\title{\large UNIVERSIDADE FEDERAL DO RIO GRANDE DO NORTE \\ INSTITUTO METRÓPOLE DIGITAL \\  BACHARELADO EM TECNOLOGIA DA INFORMAÇÃO \\ \vspace{8cm} Relatório Técnico \\  \large Projeto: Calculadora de Matrizes \vspace{5cm}}

\author{Josiluiz Nobre dos Santos \\ \large Matrícula: 20250063632}
\date{\today}

\begin{document}

   

\maketitle



\tableofcontents
\thispagestyle{empty}

\newpage

\chapter*{Introdução}
\addcontentsline{toc}{chapter}{Introdução}




    Este relatório discorre sobre a descrição de um projeto da disciplina Introdução à Técnicas de Programação, com o título: ``Calculadora de Matrizes", que relaciona o estudo dos tópicos da unidade 1 com os da unidade 2, ou seja uma melhora significativa com novas técnicas que agilizam a entrada, o pré-processamento, o processamento, a análise e a saída de dados. Uma calculadora de matrizes permite manipular os seus elementos com diversos tipos de operações, mas neste caso, para projeto são: adição, subtração, multiplicação, transporte, determinante por dois métodos distintos, matriz inversa por cofatores e um caça palavras(bônus).  








\chapter{Metodologia}


O presente projeto contou-se com o desenvolvimento nas aulas e pesquisas com o intuito de formular um código em linguagem em c, utilizando o GCC(GNU Compiler Collection) como compilador, o IDE(Ambientes de Desenvolvimento Integrado) como o VScode(Visual Studio Code) além de armazenar no Github. Dessa forma, o código foi sendo gerado a cada momento de conhecimento da linguagem e sendo reforçado por exercícios da disciplina.
O código foi desenvolvido e organizado em: Declaração de funções e alocação de matrizes, funções para realizar a operações com matrizes e parte principal ``main" que chama as funções por escolha do usuário, e com essa formação fica bem mais simples sua manutenção.


\chapter{Análise do código}
Geralmente os elementos de uma matriz são números, assim foi acrescida um ``bônus" como um caça palavras, em que a própria função cria uma matriz aleatória com letras de A a Z, e o tamanho da matriz é fornecido pelo usuário em seguida pede-se uma palavra. Como o foco do projeto eram as matrizes, então a maioria das funções possuíam repetições aninhadas como em determinante de matrizes e o caça palavras. As matrizes foram implementadas em funções percorrendo linha e em seguida coluna, onde o usuário escrevia o tamanho da matriz em seguida os seus elementos, a partir daí começavam as operações.
\\ \\
%%%%%%%%%%%%%%%%%%%%%%%%%%%%%%%%%%%%%%%%%%%%%%%%%%%%%%%%%%%%%%
Os ponteiros foram de suma importância em o todo o projeto, pois o código ficou mais eficiente no acesso de cada função em que manipulava as matrizes, sendo assim a passagem por referência de memória ajudou a alterar as variáveis iniciais, atribuindo diversos modelos, evitando cópias desnecessárias. Assim como o uso ponteiros em matrizes facilitou o seu manuseio, a alocação dinâmica permitiu reservar na memória apenas em tempo de sua execução e liberando em seguida, diferentemente em tempo de compilação o que podemos ver nesse trecho: \\ \\
Alocando por linha, reserva espaço para os endereços das linha: 
$$float **matriz = (float **)malloc(linhas * sizeof(int*)); $$
Alocando por coluna em loop: 
$$matriz[i] = (float *)malloc(colunas * sizeof(int));$$


\newpage
%%%%%%%%%%%%%%%%%%%%%%%%%%%%%%%%%%%%%%%%%%%%%%%%%%%%%%%%%%%%%%

O código foi dividido em três partes: 
\begin{itemize}
    \item [1.] Main principal: onde possui um menu para escolha de um número de 1 a 9, via comando ``switch" e acessa os distintos tipos de funções para operação com matrizes. \\ \\
* PROGRAMA SIMULANDO UMA CALCULADORA DE MATRIZES \\
* OPERAÇÕES:\\
* 1 - SOMA\\
* 2 - SUBTRAÇÃO\\
* 3 - MULTIPLICAÇÃO\\
* 4 - TRANSPORTE\\
* 5 - DETERMINANTE(MÉTODO DE LAPLACE)\\
* 6 - DETERMINANTE(MÉTODO LU)\\
* 7 - COFATOR, ADJUNTA E INVERSA \\
* 8 - CAÇA PALAVRAS \\
* 9 - SAIR 

    \item [2.] Funções que realizam as operações anteriores.
    \item [3.] Funções auxiliares que serão usadas no item anterior como para: alocar matrizes, escrever matrizes, imprimir matrizes e para desalocá-las.
\end{itemize}




\chapter{Dificuldades e soluções}

No início, a principal dificuldade foi para entender e aplicar os conceitos, até um tanto abstrato, de ponteiros e de alocação dinâmica, daí assisti mais de uma vez aulas gravadas do curso e além de pesquisas na internet. Em seguida foi a continuação da unidade 1 do projeto, isto é, para implementar novos assuntos da unidade 2, adequando com ponteiros e alocando na memória, assim os desafios ficaram cada vez mais difíceis, e isso requereu muito tempo de estudo e tentativa e erro. Um desses desafios também foi implementar o determinante de matrizes usando dois métodos distintos (Laplace e decomposição LU), em que o primeiro foi necessário usar a recursividade, pois sempre será calculado o determinante de uma matriz de ordem inferior. Já o segundo, o objetivo era manipular as linhas da matriz com operações para obter uma matriz triangular superior. \\ 
 Por fim, pensei ``como incrementar strings em matrizes?", já que estudamos mais com matrizes numéricas, daí lembrei das sugestões dadas no curso. Dessa forma, utilizei o ``caça palavra", só que escolhi a matriz de strings aleatórias do alfabeto e após digitar uma palavra, busca-se a primeira letra na matriz, se caso for encontrado, analisa as oito direções possíveis.  


\chapter*{Conclusão}
\addcontentsline{toc}{chapter}{Conclusão}



Ao decorrer do processo de aprendizado, o projeto sintetiza muito bem os assuntos estudados durante a disciplina e aponta para uma maneira prática de reunir métodos e experiências vivenciadas saindo da teoria abstrata. A evolução foi bem significativa em relação à unidade 1, pois além dos tópicos anteriores todos da unidade 2 foram contemplados de strings à alocação dinâmica. \\ Acredito que ainda dá mais para enxugar o código, principalmente em encontrar uma maneira para unir as funções de soma e subtração de matrizes, e aumentando o número de operações com matrizes estendendo a quantidade de matrizes, como por exemplo, somar 2 ou mais matrizes, além trabalhar com sistema de equações lineares.



\end{document}
